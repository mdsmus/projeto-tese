\documentclass[12pt]{article}
\usepackage{ifthen}
\usepackage[utf8,utf8x]{inputenc}
\usepackage[T1]{fontenc}
\usepackage[a4paper,margin=2cm]{geometry}
\usepackage[brazil]{babel}
\usepackage{setspace}
\usepackage{graphicx}
\usepackage{url}
\usepackage{colortbl}
\usepackage{subfig}
\usepackage{parskip}

% usar para termos estrangeiros
\newcommand{\eng}[1]{\textit{#1}}

% usar para nomes de obras
\newcommand{\opus}[1]{\textit{#1}}

% usar para nomes de termos
\newcommand{\termo}[1]{\textit{#1}}

\newcommand{\goiaba}[0]{\textit{Goiaba}}
\newcommand{\obra}[0]{\textit{Em torno da romã}}

\newcommand{\ok}{
  \multicolumn{1}{>{\columncolor[gray]{.6}}c}{}
}

\newcommand{\tri}[1]{
  #1\textsuperscript{o} t
}

\newcommand{\cabecalho}[0]{
  \textbf{\textsc{Universidade Federal da Bahia}} \\
  \textbf{\textsc{Escola de Música}} \\
  \textbf{\textsc{Programa de Pós-Graduação}} \\
  \textbf{\textsc{Doutorado em Composição}} \par
  \vspace*{1ex}
  \textbf{Orientador:} Pedro Ribeiro Kröger Júnior\\
  \textbf{Aluno: } Marcos da Silva Sampaio \\
  \textbf{Data: } \today
  \thispagestyle{empty}
}

\newcommand{\titulo}[1]{
  \vspace{1cm}
  \begin{center}{
      \Huge \textbf{Projeto de Tese} \\
    }
    \vspace{12pt}
    {\Large #1}
  \end{center}
  \vspace{1cm}
}

\setlength{\parindent}{0cm}

\begin{document}

\cabecalho
\titulo{Título do projeto}

\hyphenation{re--a--li--za--do}
\hyphenation{ne--ces--sa-ria--men--te}

\section{Introdução}
\label{sec:introducao}

%% geral:
%% remover sujeito oculto
%% remover voz passiva
%% pensar em algo sobre composição para incluir

%% no mestrado usei operações de contorno para compor
%% revisão da teoria será útil para usar operações na composição

Contornos podem ser definidos como os perfis, desenhos ou formatos de
objetos, podem ter duas ou mais dimensões, e podem relacionar altura a
comprimento, largura ou tempo. Em música contornos podem ser
abstraídos da altura, densidade, ritmo, timbre, intensidade,
etc. Contornos melódicos são abstrações do movimento de altura de
notas em função do tempo.

No estudo de contornos musicais, os valores absolutos dos elementos
são ignorados e apenas a relação de ordem de ocorrência entre estes
elementos é considerada. Cada elemento é representado por um número
inteiro, sendo que o elemento de menor valor é representado por 0, o
elemento de valor imediatamente superior é representado por 1, e assim
sucessivamente. Por exemplo, na figura \ref{fig:representacoes} o
contorno < 1 0 3 2 > ocorre como abstração de diferentes parâmetros
musicais: dinâmicas na figura \ref{fig:dynamics-in-time}, sendo que o
número 1 equivale à dinâmica \textbf{\textit{p}}, 0 à dinâmica
\textbf{\textit{ppp}}, 3 à dinâmica \textbf{\textit{ff}}, e 2 à
dinâmica \textbf{\textit{mf}}; alturas em ambas as melodias da figura
\ref{fig:melodias-1032}; e densidades de acordes na figura
\ref{fig:chord-densities-in-time}. Este contorno < 1 0 3 2 > tem seu
perfil representado graficamente na figura
\ref{fig:representacao-1032}.

\begin{figure}[h]
  \centering
  \subfloat[Representação gráfica]{
    \includegraphics{c-1032}
    \label{fig:representacao-1032}
  }
  \quad
  \subfloat[Dinâmica]{
    \includegraphics{dynamics-in-time}
    \label{fig:dynamics-in-time}
  }

  \subfloat[Altura]{
    \includegraphics{ly-1032}
    \label{fig:melodias-1032}
  }
  \subfloat[Densidade]{
    \includegraphics{chord-densities-in-time}
    \label{fig:chord-densities-in-time}
  }
  \caption{Contorno < 1 0 3 2 > como abstração de diferentes
    parâmetros}
  \label{fig:representacoes}
\end{figure}

Contornos melódicos são abordados em diversos estudos na literatura
sob o enfoque da Harmonia Tonal \cite{piston59:harmony}, Composição
\cite{schoenberg67:fundamentals,toch77:shaping}, Percepção
\cite{edworthy85:musical,dewitt.ea86:recognition}, Análise
\cite{adams76:melodic,friedmann85:methodology,friedmann87:response,marvin.ea87:relating,marvin88:generalized,marvin91:perception,marvin.ea95:generalization,morris87:composition,morris93:directions,morris95:compositional,clifford95:contour,beard03:contour,bor09:contour,schultz08:melodic,schultz09:diachronic},
e aplicação de operações de contorno na Composição Musical
\cite{sampaio08:em}.

O estudo de relações de contornos musicais é importante porque, assim
como conjuntos de notas e motivos, contornos podem ajudar a dar
coerência a uma obra musical \cite{clifford95:contour}. Além disso,
contornos são manipuláveis através de várias operações como inversão e
retrogradação, e podem ser abordados tanto do ponto de vista da
análise quanto da composição. Relações de contornos já foram usadas
satisfatoriamente para análise de obras de Mozart
\cite{beard03:contour}, Schönberg \cite{friedmann85:methodology},
Webern \cite{clifford95:contour,sampaio08:analise}, Dallapicola
\cite{marvin88:generalized}, Reich \cite{quinn97:fuzzy} e Messiaen
\cite{schultz08:melodic}, e para a composição musical
\cite{sampaio08:em}.

O estudo dessas relações de contorno tem ainda outros pontos
positivos. Contornos são estruturas de fácil representação gráfica e
experimentos mostram que leigos e músicos---estes últimos com maior
acuidade---têm habilidade de usar desenhos gráficos como o da figura
\ref{fig:representacao-1032} para representar contornos
\cite[p. 69]{marvin88:generalized}. Além disso teorias de relações de
contornos musicais fornecem conceitos e operações úteis para
comparação de contornos. Estas operações são definidas matematicamente
de forma semelhante às operações da teoria dos conjuntos (para maiores
informações sobre teoria dos conjuntos, vide
\cite{straus90:introduction}).

As teorias de relações de contornos musicais estabelecidas
inicialmente por Friedmann \cite{friedmann85:methodology}, Morris
\cite{morris87:composition,morris93:directions}, e Marvin e Laprade
\cite{marvin.ea87:relating,marvin88:generalized} contêm diversas
declarações matemáticas, como forma prima, redução de contornos e
inversão. De acordo com a equação \ref{eq:1}, a inversão de um
elemento P é a subtração da cardinalidade do contorno por 1 e pelo
valor do elemento P.

\begin{equation}
  \label{eq:1}
  I(P_n) = (q − 1 − P_n)
\end{equation}

Em estudos preliminares encontrei uma inconsistência entre o algoritmo
de forma prima e as classes de segmentos de contornos, ambos definidos
por Marvin e Laprade \cite{marvin.ea87:relating}. A tabela de classes
de segmentos de contornos (ver seção \ref{sec:fund-teor}) deveria
conter todas as formas primas possíveis de contornos de até seis
elementos. No entanto contornos como < 0 2 1 3 4 > e < 0 3 1 2 4 >,
por exemplo, estão ausentes desta tabela, embora já se encontrem em
suas devidas formas primas, de acordo com o algoritmo. Marvin
reconheceu a inconsistência em conversa via email (vide anexo). O
conceito de forma prima é importante para estas teorias, pois
estabelece relações de parentesco entre famílias de contornos. Porém
não há menção desta inconsistência em nenhum dos trabalhos posteriores
a estas definições de algoritmo de forma prima e tabela de classes de
segmentos de contornos
\cite{polansky.ea92:possible,clifford95:contour,quinn97:fuzzy,beard03:contour,bor09:contour,schultz08:melodic,schultz09:diachronic}. Este
problema de inconsistência pode implicar em fragilidades nas
ramificações destas teorias, nas análises de contornos de obras
musicais publicadas até então, e no uso futuro das operações destas
teorias na composição musical.

Eu acredito ser essencial uma revisão da totalidade de declarações
destas teorias e de análises musicais baseadas nestas declarações para
que se possa levantar outras inconsistências possíveis. Essa revisão
poderá revelar problemas nas teorias de relações de contornos musicais
e abrir caminho para propostas de soluções efetivas para tais
problemas. Desse modo esta revisão poderá representar um avanço
significativo no estado de arte das teorias de relações de contornos
musicais.

\section{Objetivos}
\label{sec:objetivos}

O principal objetivo do trabalho é levantar possíveis inconsistências
nas teorias de relações de contornos de Friedmann, Morris e Marvin, e
propor soluções para tais problemas.

Este trabalho tem como objetivos secundários:

\begin{enumerate}
\item implementar as operações das teorias de contornos no programa de
  computador \eng{MusiContour}.
\item criar uma coleção de exemplos musicais do uso de operações de
  contorno extraída da literatura de teorias de contornos.
\item criar um tutorial online para as teorias de contornos, com
  exemplos musicais explicativos.
\item criar uma versão online, livre e gratuita do programa
  \eng{MusiContour}, de cálculo de operações de relações de contornos
  musicais, para auxiliar na análise e composição musical.
\end{enumerate}

\section{Justificativa}
\label{sec:justificativa}

%% reinserir a idéia da importância da resolução do problema.
Diversos trabalhos baseados nas teorias de relações de contornos
musicais de Friedmann, Morris e Marvin podem conter fragilidades
decorrentes do problema de inconsistência entre o algoritmo de forma
prima e tabela de classes de segmento de contornos definidos por
Marvin e Laprade \cite{marvin.ea87:relating}. A revisão destas teorias
e a identificação de suas possíveis inconsistências representa um
avanço para o estado de arte de contornos musicais. Além disso o
desenvolvimento de um programa de computador para calcular essas
operações representa uma ferramenta importante para auxiliar na
análise e composição musical baseadas em relações de contornos
musicais.

\section{Fundamentação teórica}
\label{sec:fund-teor}

%% inserir bibliografia com relação ferramentas de computação x música

Contornos melódicos são abordados em diversos estudos na literatura
sob o enfoque da Harmonia Tonal \cite{piston59:harmony}, Composição
\cite{schoenberg67:fundamentals,toch77:shaping}, Percepção
\cite{edworthy85:musical,dewitt.ea86:recognition}, Análise
\cite{adams76:melodic,friedmann85:methodology,friedmann87:response,marvin.ea87:relating,marvin88:generalized,marvin91:perception,marvin.ea95:generalization,morris87:composition,morris93:directions,morris95:compositional,clifford95:contour,beard03:contour,bor09:contour,schultz08:melodic,schultz09:diachronic},
e aplicação de operações de contorno na Composição Musical
\cite{sampaio08:em}.

%% falar algumas linhas sobre cada autor

%% piston

Piston abordou funções e estruturas da melodia e a sua forma (ou
\eng{shape}) \cite{piston59:harmony}. Ele afirmou que a distribuição
de notas em uma melodia é marcada por uma variedade de mudanças de
direção, de registro, de pontos graves e agudos, e dos locais onde
estes elementos ocorrem na melodia. Ele afirmou ainda que contorno é a
reunião de todos estes elementos e que tem importância determinante do
caráter da melodia.

%% inserir schonberg, toch, edworthy e dewitt.ea

%% adams

Adams formalizou um conceito e tipologia de contornos melódicos para
análise descritiva e comparativa \cite{adams76:melodic}. Ele definiu
15 tipos de contornos a partir de três idéias: a) redução de uma
melodia a apenas quatro pontos---registro da nota inicial, final, da
nota mais aguda e mais grave; b) da relação de registro entre estes
pontos, isto é, se o ponto inicial tem mesmo registro, é mais grave ou
agudo que o ponto final, se o ponto mais agudo ocorre antes ou depois
do ponto final, e ainda se o ponto mais agudo e/ou mais grave
coincidem com os pontos inicial e/ou final; e c) da possível
recorrência entre estes pontos.

%% friedmann

Friedmann desenvolveu operações de contorno como ferramentas para
análise de música do século XX e as aplicou à análise de obras de
Schönberg \cite{friedmann85:methodology}. Ele criou operações como
\eng{Contour Adjacency Series}, \eng{Contour Adjacency Series Vector},
\eng{Contour Class}, \eng{Contour Interval}, \eng{Contour Interval
  Sucession}, \eng{Contour Interval Array}, e \eng{Contour Class
  Vector}. Friedmann ainda abordou o problema da terminologia em
operações de contorno \cite{friedmann87:response} e serviu de ponto de
partida para o trabalho de Marvin \cite{marvin88:generalized}.

%% morris

Morris abordou o \eng{design} composicional e discutiu espaços
musicais como espaço de altura, e de contorno
\cite{morris87:composition}. Ele formalizou a representação numérica
de contornos e a matriz de comparação. Expandiu ainda seu trabalho com
um algoritmo para redução de contornos \cite{morris93:directions}.

%% marvin

O trabalho de Marvin tem como ponto de partida as relações de
semelhança de espaço de contorno, definido por Morris. Ela desenvolveu
em co-autoria com Laprade \cite{marvin.ea87:relating} um algoritmo
para obtenção de forma prima, uma tabela de classes de contorno
(semelhante à tabela de Allen Forte, na teoria dos conjuntos),
estabeleceu medidas de similaridade tanto para contornos de mesma
cardinalidade quanto de cardinalidade diferente e testou estas
ferramentas em análise de obras. Além disso Marvin abordou relações
entre modelos analíticos e composicionais de estruturas musicais e a
capacidade auditiva de ouvintes \cite{marvin88:generalized}. De acordo
com ela teorias abstratas de estruturas de classes de alturas e
conjuntos não refletem a percepção auditiva dos ouvintes tão bem
quanto teorias que modelam a articulação destas estruturas de classes
de alturas e conjuntos \cite[p. 228]{marvin88:generalized}. Ouvintes
costumam criar representações gráficas do contexto musical, como
mudanças de direção da melodia, padrões de duração relativa e assim
por diante. Estas representações podem ser comparadas umas às outras a
partir da aplicação e generalização das teorias de contornos de Morris
\cite[p. 229]{marvin88:generalized}. Marvin propôs ainda generalização
de contornos para outros domínios e aplicou esta idéia ao espaçamento
de acordes na obra pianística de Luigi Dallapicola. Marvin propôs como
trabalho futuro na área de psicologia musical experimentos que
apliquem a sua precisa definição de equivalência de contornos. Por fim
propôs um currículo para pedagogia de teoria para música não tonal.

%% inserir polansky

%% quinn

Quinn \cite{quinn97:fuzzy} combinou usos lógica difusa\footnote{Lógica
  difusa é uma extensão da lógica booleana que admite valores lógicos
  intermediários entre o verdadeiro e o falso. Esse tipo de lógica
  lida com assertivas como ``5 é mais próximo de 4 do que 7''.} com
teoria de contornos. O ponto de partida do trabalho é um modelo de
audição em que o ouvinte foca no grau de semelhança de contornos entre
as repetições de uma melodia e aquilo que já foi anteriormente
percebido. Quinn utilizou ferramentas de comparação de contornos para
comparar cada repetição com a média ponderada das antecessoras. Ele
testou esta teoria na obra de Steve Reich, \opus{Desert Music}.

%% clifford

Clifford investigou o uso de relações de contorno como elemento
estrutural em obras pré-seriais de Anton Webern
\cite{clifford95:contour}. Concluiu em seu trabalho que na ausência de
outros elementos de organização de alturas, contorno representa um
fator estrutural igual em significado a relações de alturas ou de
classes de conjuntos. Verificou a existência de três níveis de
contornos: a) contorno no nível melódico, b) contorno como delineador
da forma, que ocorre em larga escala, e c) contorno textural, que
envolve contornos distribuído em vozes diferentes. Além disso propôs
trabalhos futuros como investigação de contornos em outras obras de
Webern, exploração do contorno textural em múltiplos níveis, relações
entre percepção auditiva e imagens dos contornos, e por fim exploração
de contornos associados a outros parâmetros além de altura.

%% beard

Beard implementou a duração na análise a partir de contornos melódicos
através do princípio estatístico de múltiplas regressões lineares
\cite{beard03:contour}. Ele aplicou com êxito este princípio na
análise dos primeiros temas de 19 sonatas de Mozart. Beard sugere como
trabalho futuro a análise de estruturas de outras partes das obras de
Mozart ou ainda das obras de outros compositores, experimentos com as
técnicas de regressão e estudos de percepção auditiva para investigar
a relação entre percepção e regressão.

%% bor

Bor \cite{bor09:contour} desenvolveu um conjunto de algoritmos de
redução de contornos a partir tomando como ponto de partida o
algoritmo de redução de Morris \cite{morris93:directions}. Os
\eng{window algorithms}, de Bor, têm como principal característica
considerar um número específico de elementos para redução. Ele
utilizou a analogia de uma janela de um trem em movimento, que tem uma
largura fixa e através da qual percebe-se apenas uma porção da
paisagem. A aplicação destes algoritmos é demonstrada em vários
exemplos da literatura de música do século XX. Bor ainda explorou
implicações fenomenológicas e cognitivas, e introduz a extensão da
teoria para o domínio da duração.

%% schultz

Schultz \cite{schultz09:diachronic} desenvolveu uma teoria de relações
de contorno diacrônico-transformacional. Seu objetivo foi sistematizar
o modo como todos os contornos se relacionam uns com os outros com
base em uma visão transformacional em tempo real através do uso de
estruturas de árvores hierárquicas. Além disso ele teve como objetivo
tentar mudar a perspectiva tradicional do analista/ouvinte da posição
de observador passivo para um participante ativo.

%% minha dissertação

Em minha dissertação \cite{sampaio08:em} experimentei funcionalidades
das operações de contorno usadas para análise na composição da obra
\obra{}. Compus a maior parte do material da obra a partir de
operações aplicadas a um único contorno. Embora tenha concentrado o
estudo em contornos associados à altura, experimentei associações com
andamento, densidade e textura. Além disso iniciei o desenvolvimento
do programa de computador \goiaba{}, para processamento de contornos.

As descrições das teorias de contornos nem sempre são feitas pelos
seus autores de uma forma totalmente clara. Por exemplo, o algoritmo
de forma prima de Marvin e Laprade \cite{marvin.ea87:relating} é
definido formalmente e ilustrado com exemplos musicais, porém não há
informações que esclareçam o porquê dos passos do algoritmo. Por
exemplo, a inconsistência mencionada na introdução ocorre no segundo
passo do algoritmo (vide equação \ref{eq:2}). Não há uma explicação
musical do que este passo significa. Talvez se houvesse uma
preocupação com o entendimento do leitor, esta inconsistência não
existisse.

\begin{equation}
  \label{eq:2}
  If\ (n-1) - cp(n) < cp(1),\ then\ invert\ the\ cseg
\end{equation}

A implementação de uma teoria em um programa de computador é um
exercício poderoso no processo de aprendizagem, pois o programador é
obrigado a expressar de forma precisa a compreensão que tem de uma
teoria. Dessa forma a implementação das operações de contorno em um
programa de computador ajuda na compreensão da teoria, uma vez que é
preciso entender totalmente como cada operação se processa e que
funcionalidade tem para que se possa implementá-las
\cite{sampaio08:em}.

Programas de computador também podem ajudar a encontrar erros de
difícil percepção manual. Um exemplo prático desta aplicação na área
de Música é o desenvolvimento do programa Rameau\footnote{Disponível
  em \url{github.com/kroger/rameau}.}
\cite{kroger08:rameau,passos.ea09:functional}. Na implementação deste
programa, os 371 corais de J.S. Bach foram convertidos do formato MIDI
para o formato Lilypond\footnote{Vide Lilypond em
  \url{lilypond.org}}. Como o formato MIDI não guarda informações de
enarmonia \cite{selfridge-field97:beyond}, foi necessário revisar
todas as notas de todos os corais. Durante o desenvolvimento de
aplicações em Musicologia Computacional para os corais de Bach
\cite{kroger08:musicologia}, os desenvolvedores programaram
ferramentas para detectar automaticamente estes erros de enarmonia.

\section{Procedimentos metodológicos}
\label{sec:metodologia}

O levantamento das inconsistências mencionadas pode ter precisão e
eficácia elevadas por meio do uso de ferramentas computacionais. Por
esta razão este trabalho demanda do desenvolvimento de um programa de
computador desenhado para abrigar funções e testes para todas as
definições das teorias de relações de contornos musicais de Friedmann,
Morris e Marvin, e das suas ramificações.

Tal programa deve ser implementado segundo uma abordagem \eng{bottom
  up} \cite{graham94:lisp} de forma que possa ser desenvolvido e
expandido à medida que se avança na programação das definições das
teorias de contornos musicais, e que se analiza os resultados.

Este trabalho de levantamento de inconsistências nas teorias de
contornos musicais será realizado por meio das seguintes ações:

\begin{enumerate}
\item Análise de relações de contornos em obras da literatura. Esta
  etapa possibilita um maior entendimento do funcionamento das
  operações das teorias por meio da análise de como essas relações de
  contorno ocorrem na literatura.
\item Composição de experimentos baseados em relações de contornos.
  Esta etapa possibilita um maior entendimento do funcionamento das
  operações das teorias por meio da experimentação composicional.
\item Levantamento e catalogação de todas as operações de contorno
  presentes na literatura de contornos musicais. Esta etapa é manual e
  será feito durante a revisão da literatura.
\item Levantamento de todas as operações de contorno utilizadas nas
  análises musicais baseadas nas teorias de contornos musicais. Esta
  etapa também é manual e será feita durante a revisão de literatura.
\item Catalogação dos exemplos ilustrativos das operações de contorno
  presentes na literatura de contornos musicais. Esta etapa consiste
  na digitalização dos exemplos.
\item Implementação das operações de contornos em algoritmos.
\item Implementação de testes funcionais com os exemplos ilustrativos
  das teorias. Estes testes funcionais retornam erro quando alguma
  função programada tem resultado diferente do esperado. A deteção
  destes erros permite correções do programa e identificação de
  inconsistências das teorias.
\item Implementação de rotinas de verificação. Esta etapa, de
  abordagem completamente \eng{bottom up} consiste em criar funções
  simples que ajudam na visualização de problemas com as operações
  programadas.
\item Verificação e análise permanente de resultados.
\item Avaliação de possíveis soluções para problemas.
\end{enumerate}

É importante destacar que a maioria destas ações são implementadas
simultaneamente, não havendo necessariamente uma hierarquia de
pré-requisitos entre elas.

\section{Resultados esperados}
\label{sec:resultados-esperados}

Espera-se com este trabalho levantar e propor soluções para as
possíveis inconsistências nas definições teorias de contornos
musicais, criar um catálogo de exemplos de uso de operações de
contorno baseado nos exemplos da literatura, e desenvolver o programa
de computador online \eng{MusiContour}, de cálculo de operações de
contornos musicais, para auxiliar na análise e composição musical.

\section{Estrutura de tópicos da tese}
\label{sec:estrutura-de-topicos}

%% explicar o que é cada ítem
\begin{enumerate}
\item Introdução
\item Revisão de literatura (comparação de teorias e sondagem do
  estado de arte)
\item Apresentação das ferramentas computacionais
\item Discussão de como as ferramentas ajudam a encontrar os problemas
  das teorias de contornos
\item Propostas para solução de problemas das teorias de contornos
\item Conclusões
\end{enumerate}

\section{Cronograma}
\label{sec:cronograma}

O cronograma deste projeto é retroativo ao início do curso do
doutorado, pois há atividades necessárias já cumpridas neste
período. Estas atividades incluem:

\begin{enumerate}
\item Créditos das disciplinas do Programa de Pós-graduação em música
\item Estudo de programação em computador (linguagens
  Lisp\footnote{Vide programação do \textit{Goiaba} em
    \url{github.com/mdsmus/goiaba}.} e Python\footnote{Vide
    programação do \textit{MusiContour} em
    \url{github.com/mdsmus/MusiContour}.})
\item Composição de cinco obras:
\begin{enumerate}
\item Nanopeça II (miniatura para piano\footnote{Vide
    \url{topo-base.composicao-e-cultura.com/artigos/22/devaneio-01-de-baixo-pra-cima}.})
\item Nanopeça III (miniatura para piano\footnote{Vide
    \url{topo-base.composicao-e-cultura.com/artigos/56/devaneio-02-seguindo-a-receita-do-bolo}.})
\item Espiral\footnote{Vide
  \url{semcompciclo.wordpress.com/2009/12/29/espiral-op-7-um-relato/}.}
\item Fuxico\footnote{Vide
    \url{marcosdisilva.net/pt/obras-op-9.html}.}
\item Genética\footnote{Vide
    \url{topo-base.composicao-e-cultura.com/artigos/110/genetica-op8}.},
  composta inteiramente baseada em contornos, da forma à escolha das
  alturas.
\end{enumerate}
\item Análise de contornos da \opus{Sonata para piano op.2} de
  L.V.Beethoven, do \opus{Contrapunctus VI}, da \opus{Arte da Fuga},
  de J.S.Bach, e do \opus{Fünf Sätze für Streichquartett}, mov.II, de
  Anton Webern.
\item Levantamento e implementação das operações definidas por
  Friedmann \cite{friedmann85:methodology}, Morris
  \cite{morris93:directions} e Marvin
  \cite{marvin.ea87:relating}\footnote{Vide operações implementadas no
    MusiContour, em \url{genos.mus.br/MusiContour}.} e de seus testes
  funcionais:
  \begin{enumerate}
  \item \eng{translation (normal form)}
  \item \eng{prime form}
  \item \eng{comparison matrix}
  \item \eng{internal diagonals (CAS)}
  \item \eng{contour adjacency series vector}
  \item \eng{contour interval succession}
  \item \eng{contour interval array}
  \item \eng{contour class vector I e II}
  \item \eng{contour subsets}
  \item \eng{contour segment classes table}
  \item \eng{contour similarity}
  \item \eng{all embed contour}
  \item \eng{Morris contour reduction algorithm}
  \end{enumerate}
\end{enumerate}

Além destas tarefas já cumpridas, o cronograma inclui as atividades
listadas na seção \ref{sec:metodologia} e as obrigações relacionadas
ao Programa de Pós-graduação, como a realização do recital com obras
compostas no período do doutorado.

\begin{tabular}{|l||c|c|c|c||c|c|c|c||c|c|c|c||c|c|c|c||}
  \hline
  & \multicolumn{4}{|c||}{2009} & \multicolumn{4}{|c||}{2010} & \multicolumn{4}{|c||}{2011} & \multicolumn{4}{|c||}{2012} \\
  \hline
  \hline
  \textbf{Atividades}
  & 1 & 2 & 3 & 4 & 1 & 2 & 3 & 4 & 1 & 2 & 3 & 4 & 1 & 2 & 3 & 4 \\
  \hline
  \hline
  % modelo & & & & & & & & & & & & & & & & \\
  Estudo de programação & \ok & \ok & \ok & \ok & \ok & \ok & \ok &
  \ok & \ok & \ok & \ok & \ok & \ok & & & \\
  Créditos PPGMUS & \ok & \ok & \ok & \ok & \ok & \ok & \ok & \ok & & & & & & & & \\
  Análise de obras da literatura & & & & & & \ok & \ok & & & & & & & & & \\
  Composição de experimentos & & \ok & & \ok & & \ok & \ok & & \ok & & & & & & & \\
  Recital com obras próprias & & & & & & & & & & & \ok & & & & & \\
  Levant. de operações (teoria) & & & & & & \ok & & \ok & \ok & \ok & & & & & & \\
  Levant. de operações (análises) & & & & & & & & & \ok & \ok & \ok & \ok & & & & \\
  Catalogação de exemplos & & & & & & & & & \ok & \ok & \ok & \ok & & & & \\
  Implem. de algoritmos & & & & & & \ok & & \ok & \ok & \ok & \ok & \ok & & & & \\
  Implem. de testes funcionais & & & & & & \ok & & \ok & \ok & \ok & \ok & \ok & & & & \\
  Implem. de rotinas de verificação & & & & & & & & & & & \ok & \ok & & & & \\
  Verificação e análise de resultados & & & & & & & & & & & \ok & \ok & & & & \\
  Avaliação de sugestões & & & & & & & & & & & & & \ok & \ok & & \\
  Escrita da tese & & & & & & & & & & & & & & \ok & \ok & \ok \\
  \hline
\end{tabular}

\nocite{marvin.ea95:generalization,sampaio.ea10:villa-lobos,sampaio.ea09:goiaba,sampaio.ea09:musical}

\pagebreak
\renewcommand{\refname}{Bibliografia}
\bibliography{bib,genos}
\bibliographystyle{plain}

\end{document}