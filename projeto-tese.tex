\documentclass[12pt]{article}
\usepackage{ifthen}
\usepackage[utf8,utf8x]{inputenc}
\usepackage[a4paper,margin=2cm]{geometry}
\usepackage[brazil]{babel}
\usepackage{setspace}
\usepackage{graphicx}
\usepackage{url}
\usepackage{colortbl}

% usar para termos estrangeiros
\newcommand{\eng}[1]{\textit{#1}}

% usar para nomes de obras
\newcommand{\opus}[1]{\textit{#1}}

% usar para nomes de termos
\newcommand{\termo}[1]{\textit{#1}}

\newcommand{\ok}{
  \multicolumn{1}{>{\columncolor[gray]{.6}}c}{}
}

\newcommand{\tri}[1]{
  #1\textsuperscript{o} t
}

\newcommand{\cabecalho}[0]{
  \textbf{\textsc{Universidade Federal da Bahia}} \\
  \textbf{\textsc{Escola de Música}} \\
  \textbf{\textsc{Programa de Pós-Graduação}} \\
  \textbf{\textsc{Doutorado em Composição}} \par
  \vspace*{1ex}
  \textbf{Orientador:} Pedro Ribeiro Kröger Júnior\\
  \textbf{Aluno: } Marcos da Silva Sampaio \\
  \textbf{Data: } \today
  \thispagestyle{empty}
}

\newcommand{\titulo}[1]{
  \vspace{1cm}
  \begin{center}{
      \Huge \textbf{Projeto de Tese} \\
    }
    \vspace{12pt}
    {\Large #1}
  \end{center}
  \vspace{1cm}
}

\setlength{\parindent}{0cm}

\begin{document}

\cabecalho
\titulo{Título do projeto}

\section{Introdução}
\label{sec:introducao}

As teorias de contornos musicais definidas por Friedmann
\cite{friedmann85:methodology}, Morris
\cite{morris87:composition,morris93:directions}, Marvin e Laprade
\cite{marvin.ea87:relating,marvin88:generalized} contêm diversas
declarações matemáticas, como inversão (vide equação
(\ref{eq:1})\footnote{De acordo com esta equação, a inversão de um
  elemento P é a subtração da cardinalidade do contorno por 1 e pelo
  valor do elemento P.}), forma prima e redução de contornos.

\begin{equation}
  \label{eq:1}
  I(P_n) = (q − 1 − P_n)
\end{equation}

Estudos preliminares mostram uma inconsistência entre o algoritmo de
forma prima e as classes de segmentos de contornos definidas por
Marvin e Laprade \cite{marvin.ea87:relating}.

O trabalho inicial destes autores tem sido expandido desde então por
Polansky e Bassein \cite{polansky.ea92:possible}, Clifford
\cite{clifford95:contour}, Quinn \cite{quinn97:fuzzy}, Beard
\cite{beard03:contour}, Bor \cite{bor09:contour} e Schultz
\cite{schultz08:melodic,schultz09:diachronic}. Nenhum destes autores,
entretanto, menciona esta inconsistência.

O conceito de forma prima é bastante importante para estas teorias,
pois estabelece relações entre famílias de contornos. A forma prima é,
então, um pilar das teorias de contornos musicais. O problema de
inconsistência relacionado com a definição de forma prima pode
implicar em fragilidades nas ramificações destas teorias, e nas
análises de contornos de obras musicais publicadas até então.

É necessária uma revisão de todas as declarações destas teorias e suas
ramificações para levantar possíveis inconsistências e propor soluções.

\section{Objetivos}
\label{sec:objetivos}

O principal objetivo do trabalho é levantar possíveis inconsistências
nas teorias de contornos de Friedmann, Morris e Marvin e propor
soluções para tais problemas.

Este trabalho tem como objetivos secundários:

\begin{enumerate}
\item implementar as operações das teorias de contornos no
  \eng{Villa-Lobos Contour Module}\footnote{Disponível em
    \url{genos.mus.br/villa-lobos/contour-module}}.
\end{enumerate}

\section{Justificativa}
\label{sec:justificativa}

Diversos trabalhos baseados nas teorias de contornos musicais de
Friedmann, Morris e Marvin, podem conter fragilidades decorrentes do
problema de inconsistência entre o algoritmo de forma prima e tabela
de classes de segmento de contornos de Marvin e Laprade
\cite{marvin.ea87:relating}.

\section{Fundamentação teórica}
\label{sec:fund-teor}

\section{Procedimentos metodológicos}
\label{sec:metodologia}

O levantamento das inconsistências pode ter precisão e eficácia
elevados com o uso de ferramentas computacionais. Por esta razão este
trabalho demanda do desenvolvimento de um programa de computador
desenhado para abrigar funções e testes para todas as definições das
teorias de contorno musical de Friedmann, Morris e Marvin, e das suas
ramificações.

Tal programa deve ser desenvolvido segundo uma abordagem \eng{bottom
  up} \cite{graham94:lisp} de forma que possa ser desenvolvido e
expandido à medida que se avança na programação das definições das
teorias de contornos musicais.

\section{Resultados esperados}
\label{sec:resultados-esperados}

Espera-se com este trabalho levantar e propor soluções para as
possíveis fragilidades nas definições teorias de contornos musicais e

\section{Estrutura de tópicos da tese}
\label{sec:estrutura-de-topicos}

\begin{itemize}
\item Introdução
\item Revisão de literatura (comparação de teorias e sondagem do
  estado de arte)
\item Apresentação das ferramentas computacionais
\item Discussão de como as ferramentas ajudam a encontrar os problemas
  das teorias de contornos.
\item Propostas para solução de problemas das teorias de contornos.
\end{itemize}

\section{Cronograma}
\label{sec:cronograma}

\begin{table}[htbp]
  \small
  \centering
  \begin{tabular}{|l||c|c|c|c||c|c|c|c|}
    \hline
    & \multicolumn{4}{|c||}{2011} & \multicolumn{4}{|c||}{2012} \\
    \hline
    \hline
    \textbf{Atividades}
  \end{tabular}
  \caption{Cronograma das atividades (em trimestres)}
  \label{tab:cronograma}
\end{table}

\nocite{marvin.ea95:generalization}

\renewcommand{\refname}{Bibliografia}
\bibliography{bib}
\bibliographystyle{plain}

\end{document}