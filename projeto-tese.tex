\documentclass[12pt]{article}
\usepackage{ifthen}
\usepackage[utf8,utf8x]{inputenc}
\usepackage[T1]{fontenc}
\usepackage[a4paper,margin=2cm]{geometry}
\usepackage[brazil]{babel}
\usepackage{setspace}
\usepackage{graphicx}
\usepackage{url}
\usepackage{colortbl}
\usepackage{subfig}

% usar para termos estrangeiros
\newcommand{\eng}[1]{\textit{#1}}

% usar para nomes de obras
\newcommand{\opus}[1]{\textit{#1}}

% usar para nomes de termos
\newcommand{\termo}[1]{\textit{#1}}

\newcommand{\ok}{
  \multicolumn{1}{>{\columncolor[gray]{.6}}c}{}
}

\newcommand{\tri}[1]{
  #1\textsuperscript{o} t
}

\newcommand{\cabecalho}[0]{
  \textbf{\textsc{Universidade Federal da Bahia}} \\
  \textbf{\textsc{Escola de Música}} \\
  \textbf{\textsc{Programa de Pós-Graduação}} \\
  \textbf{\textsc{Doutorado em Composição}} \par
  \vspace*{1ex}
  \textbf{Orientador:} Pedro Ribeiro Kröger Júnior\\
  \textbf{Aluno: } Marcos da Silva Sampaio \\
  \textbf{Data: } \today
  \thispagestyle{empty}
}

\newcommand{\titulo}[1]{
  \vspace{1cm}
  \begin{center}{
      \Huge \textbf{Projeto de Tese} \\
    }
    \vspace{12pt}
    {\Large #1}
  \end{center}
  \vspace{1cm}
}

\setlength{\parindent}{0cm}

\begin{document}

\cabecalho
\titulo{Título do projeto}

\section{Introdução}
\label{sec:introducao}

Contornos podem ser definidos como os perfis, desenhos ou formatos de
objetos. Podem ser bidimensionais e associar altura a comprimento,
largura ou tempo. Em música contornos podem ser associados a altura,
densidade, ritmo, homogeneidade/heterogeneidade orquestral,
intensidade, etc. Contornos melódicos estão relacionados com movimento
de altura de notas em função do tempo.

Contornos podem ser representados com números que refletem o perfil de
uma estrutura musical. Por exemplo, a figura \ref{fig:representacoes}
mostra o contorno < 1 0 3 2 > associado a diferentes parâmetros
musicais. Neste exemplo o número 0 representa o menor ponto e a
diferença entre dois pontos é 1. Por exemplo, na figura
\ref{fig:dynamics-in-time} o número 1 equivale à dinâmica
\textbf{\textit{p}}, 0 à dinâmica \textbf{\textit{ppp}}, 3 à dinâmica
\textbf{\textit{ff}}, e 2 à dinâmica \textbf{\textit{mf}}. Na figura
\ref{fig:melodias-1032} ambas as melodias estão relacionadas ao
contorno < 1 0 3 2 >. A figura \ref{fig:chord-densities-in-time} mostra
acordes com densidades relacionadas ao contorno < 1 0 3 2 >. O contorno
< 1 0 3 2 > tem seu perfil representado graficamente pela figura
\ref{fig:representacao-1032}.

\begin{figure}[h]
  \centering
  \subfloat[Representação gráfica]{
    \includegraphics{c-1032}
    \label{fig:representacao-1032}
  }
  \quad
  \subfloat[Dinâmica]{
    \includegraphics{dynamics-in-time}
    \label{fig:dynamics-in-time}
  }

  \subfloat[Altura]{
    \includegraphics{ly-1032}
    \label{fig:melodias-1032}
  }
  \subfloat[Densidade]{
    \includegraphics{chord-densities-in-time}
    \label{fig:chord-densities-in-time}
  }
  \caption{Contorno < 1 0 3 2 > associado a diferentes parâmetros}
  \label{fig:representacoes}
\end{figure}

O estudo de contornos é importante porque, assim como conjuntos de
notas e motivos, contornos podem ajudar a dar coerência a uma obra
musical \cite{clifford95:contour}. Contornos representam estruturas
musicais manipuláveis através de várias operações como inversão e
retrogradação, e podem ser abordados tanto do ponto de vista da
análise quanto da composição. Contornos já foram usados
satisfatoriamente para analisar obras de Mozart
\cite{beard03:contour}, Schönberg \cite{friedmann85:methodology},
Webern \cite{clifford95:contour,sampaio08:analise}, Dallapicola
\cite{marvin88:generalized} e Reich \cite{quinn97:fuzzy}, e para a
composição musical \cite{sampaio08:em}.

O estudo de contornos tem ainda outros pontos positivos
significativos. Contornos são estruturas de fácil representação
gráfica comprovada por experimentos que mostram que leigos e
músicos---estes últimos com maior acuidade---têm habilidade de usar
desenhos gráficos como o da figura \ref{fig:representacao-1032} para
representar contornos \cite[p. 69]{marvin88:generalized}. Além disso
teorias de contornos fornecem conceitos e operações úteis para
comparação de contornos. Estas operações são definidas matematicamente
de forma semelhante às operações da teoria dos conjuntos.

As teorias de contornos musicais definidas por Friedmann
\cite{friedmann85:methodology}, Morris
\cite{morris87:composition,morris93:directions}, Marvin e Laprade
\cite{marvin.ea87:relating,marvin88:generalized} contêm diversas
declarações matemáticas, como inversão (vide equação
(\ref{eq:1})\footnote{De acordo com esta equação, a inversão de um
  elemento P é a subtração da cardinalidade do contorno por 1 e pelo
  valor do elemento P.}), forma prima e redução de contornos.

\begin{equation}
  \label{eq:1}
  I(P_n) = (q − 1 − P_n)
\end{equation}

Estudos preliminares mostram uma inconsistência entre o algoritmo de
forma prima e as classes de segmentos de contornos definidas por
Marvin e Laprade \cite{marvin.ea87:relating}.

O trabalho inicial destes autores tem sido expandido desde então por
Polansky e Bassein \cite{polansky.ea92:possible}, Clifford
\cite{clifford95:contour}, Quinn \cite{quinn97:fuzzy}, Beard
\cite{beard03:contour}, Bor \cite{bor09:contour} e Schultz
\cite{schultz08:melodic,schultz09:diachronic}. Nenhum destes autores,
entretanto, menciona esta inconsistência.

O conceito de forma prima é bastante importante para estas teorias,
pois estabelece relações entre famílias de contornos. A forma prima é,
então, um pilar das teorias de contornos musicais. O problema de
inconsistência relacionado com a definição de forma prima pode
implicar em fragilidades nas ramificações destas teorias, e nas
análises de contornos de obras musicais publicadas até então.

É necessária uma revisão de todas as declarações destas teorias e suas
ramificações para levantar possíveis inconsistências e propor soluções.

\section{Objetivos}
\label{sec:objetivos}

O principal objetivo do trabalho é levantar possíveis inconsistências
nas teorias de contornos de Friedmann, Morris e Marvin e propor
soluções para tais problemas.

Este trabalho tem como objetivos secundários:

\begin{enumerate}
\item implementar as operações das teorias de contornos no
  \eng{Villa-Lobos Contour Module}\footnote{Disponível em
    \url{genos.mus.br/villa-lobos/contour-module}}.
\end{enumerate}

\section{Justificativa}
\label{sec:justificativa}

Diversos trabalhos baseados nas teorias de contornos musicais de
Friedmann, Morris e Marvin, podem conter fragilidades decorrentes do
problema de inconsistência entre o algoritmo de forma prima e tabela
de classes de segmento de contornos de Marvin e Laprade
\cite{marvin.ea87:relating}.

\section{Fundamentação teórica}
\label{sec:fund-teor}

\section{Procedimentos metodológicos}
\label{sec:metodologia}

O levantamento das inconsistências pode ter precisão e eficácia
elevados com o uso de ferramentas computacionais. Por esta razão este
trabalho demanda do desenvolvimento de um programa de computador
desenhado para abrigar funções e testes para todas as definições das
teorias de contorno musical de Friedmann, Morris e Marvin, e das suas
ramificações.

Tal programa deve ser desenvolvido segundo uma abordagem \eng{bottom
  up} \cite{graham94:lisp} de forma que possa ser desenvolvido e
expandido à medida que se avança na programação das definições das
teorias de contornos musicais.

\section{Resultados esperados}
\label{sec:resultados-esperados}

Espera-se com este trabalho levantar e propor soluções para as
possíveis fragilidades nas definições teorias de contornos musicais e

\section{Estrutura de tópicos da tese}
\label{sec:estrutura-de-topicos}

\begin{itemize}
\item Introdução
\item Revisão de literatura (comparação de teorias e sondagem do
  estado de arte)
\item Apresentação das ferramentas computacionais
\item Discussão de como as ferramentas ajudam a encontrar os problemas
  das teorias de contornos.
\item Propostas para solução de problemas das teorias de contornos.
\end{itemize}

\section{Cronograma}
\label{sec:cronograma}

\begin{table}[htbp]
  \small
  \centering
  \begin{tabular}{|l||c|c|c|c||c|c|c|c|}
    \hline
    & \multicolumn{4}{|c||}{2011} & \multicolumn{4}{|c||}{2012} \\
    \hline
    \hline
    \textbf{Atividades}
  \end{tabular}
  \caption{Cronograma das atividades (em trimestres)}
  \label{tab:cronograma}
\end{table}

\nocite{marvin.ea95:generalization}

\renewcommand{\refname}{Bibliografia}
\bibliography{bib}
\bibliographystyle{plain}

\end{document}