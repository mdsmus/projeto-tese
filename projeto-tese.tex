\documentclass[12pt]{article}
\usepackage{ifthen}
\usepackage[utf8,utf8x]{inputenc}
\usepackage[a4paper,margin=2cm]{geometry}
\usepackage[brazil]{babel}
\usepackage{setspace}
\usepackage{graphicx}
\usepackage{url}
\usepackage{colortbl}

% usar para termos estrangeiros
\newcommand{\eng}[1]{\textit{#1}}

% usar para nomes de obras
\newcommand{\opus}[1]{\textit{#1}}

% usar para nomes de termos
\newcommand{\termo}[1]{\textit{#1}}

\newcommand{\ok}{
  \multicolumn{1}{>{\columncolor[gray]{.6}}c}{}
}

\newcommand{\tri}[1]{
  #1\textsuperscript{o} t
}

\newcommand{\cabecalho}[0]{
  \textbf{\textsc{Universidade Federal da Bahia}} \\
  \textbf{\textsc{Escola de Música}} \\
  \textbf{\textsc{Programa de Pós-Graduação}} \\
  \textbf{\textsc{Doutorado em Composição}} \par
  \vspace*{1ex}
  \textbf{Orientador:} Pedro Ribeiro Kröger Júnior\\
  \textbf{Aluno: } Marcos da Silva Sampaio \\
  \textbf{Data: } \today
  \thispagestyle{empty}
}

\newcommand{\titulo}[1]{
  \vspace{1cm}
  \begin{center}{
      \Huge \textbf{Projeto de Tese} \\
    }
    \vspace{12pt}
    {\Large #1}
  \end{center}
  \vspace{1cm}
}

\setlength{\parindent}{0cm}

\begin{document}

\cabecalho
\titulo{Título do projeto}

\section{Introdução}
\label{sec:introducao}

Algumas definições das teorias de contornos contêm erros. Por exemplo,
a tabela de Marvin e Laprade \cite{marvin.ea87:relating}.

\section{Objetivos}
\label{sec:objetivos}

Rastrear erros das teorias de contorno e propor soluções para
preencher tais lacunas.

\section{Justificativa}
\label{sec:justificativa}

Há alguns trabalhos baseados nestas teorias (pelo menos 4 teses) e não
há menção a estes erros.

\section{Metodologia}
\label{sec:metodologia}

Programação de algoritmos para todas as definições/funções presentes
na literatura e realização de testes com os exemplos da literatura.

\section{Resultados esperados}
\label{sec:resultados-esperados}

Encontrar outros erros nas definições e possíveis soluções para alguns
desses erros.

\section{Estrutura de tópicos da tese}
\label{sec:estrutura-de-topicos}

\begin{itemize}
\item Introdução
\item Revisão de literatura (comparação de teorias e sondagem do
  estado de arte)
\item Apresentação das ferramentas computacionais
\item Discussão de como as ferramentas ajudam a encontrar os problemas
  das teorias de contornos.
\item Propostas para solução de problemas das teorias de contornos.
\end{itemize}

\section{Cronograma}
\label{sec:cronograma}

\begin{table}[htbp]
  \small
  \centering
  \begin{tabular}{|l||c|c|c|c||c|c|c|c|}
    \hline
    & \multicolumn{4}{|c||}{2011} & \multicolumn{4}{|c||}{2012} \\
    \hline
    \hline
    \textbf{Atividades}
  \end{tabular}
  \caption{Cronograma das atividades (em trimestres)}
  \label{tab:cronograma}
\end{table}

\renewcommand{\refname}{Bibliografia}
\bibliography{bib}
\bibliographystyle{plain}

\end{document}