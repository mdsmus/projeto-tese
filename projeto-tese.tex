\documentclass[12pt]{article}
\usepackage{ifthen}
\usepackage[utf8,utf8x]{inputenc}
\usepackage[T1]{fontenc}
\usepackage[a4paper,margin=2cm]{geometry}
\usepackage[brazil]{babel}
\usepackage{setspace}
\usepackage{graphicx}
\usepackage{url}
\usepackage{colortbl}
\usepackage{subfig}

% usar para termos estrangeiros
\newcommand{\eng}[1]{\textit{#1}}

% usar para nomes de obras
\newcommand{\opus}[1]{\textit{#1}}

% usar para nomes de termos
\newcommand{\termo}[1]{\textit{#1}}

\newcommand{\goiaba}[0]{\textit{Goiaba}}
\newcommand{\obra}[0]{\textit{Em torno da romã}}

\newcommand{\ok}{
  \multicolumn{1}{>{\columncolor[gray]{.6}}c}{}
}

\newcommand{\tri}[1]{
  #1\textsuperscript{o} t
}

\newcommand{\cabecalho}[0]{
  \textbf{\textsc{Universidade Federal da Bahia}} \\
  \textbf{\textsc{Escola de Música}} \\
  \textbf{\textsc{Programa de Pós-Graduação}} \\
  \textbf{\textsc{Doutorado em Composição}} \par
  \vspace*{1ex}
  \textbf{Orientador:} Pedro Ribeiro Kröger Júnior\\
  \textbf{Aluno: } Marcos da Silva Sampaio \\
  \textbf{Data: } \today
  \thispagestyle{empty}
}

\newcommand{\titulo}[1]{
  \vspace{1cm}
  \begin{center}{
      \Huge \textbf{Projeto de Tese} \\
    }
    \vspace{12pt}
    {\Large #1}
  \end{center}
  \vspace{1cm}
}

\setlength{\parindent}{0cm}

\begin{document}

\cabecalho
\titulo{Título do projeto}

\section{Introdução}
\label{sec:introducao}

Contornos podem ser definidos como os perfis, desenhos ou formatos de
objetos. Podem ser bidimensionais e associar altura a comprimento,
largura ou tempo. Em música contornos podem ser associados a altura,
densidade, ritmo, homogeneidade/heterogeneidade orquestral,
intensidade, etc. Contornos melódicos estão relacionados com movimento
de altura de notas em função do tempo.

Contornos podem ser representados com números que refletem o perfil de
uma estrutura musical. Por exemplo, a figura \ref{fig:representacoes}
mostra o contorno < 1 0 3 2 > associado a diferentes parâmetros
musicais. Neste exemplo o número 0 representa o menor ponto e a
diferença entre dois pontos é 1. Por exemplo, na figura
\ref{fig:dynamics-in-time} o número 1 equivale à dinâmica
\textbf{\textit{p}}, 0 à dinâmica \textbf{\textit{ppp}}, 3 à dinâmica
\textbf{\textit{ff}}, e 2 à dinâmica \textbf{\textit{mf}}. Na figura
\ref{fig:melodias-1032} ambas as melodias estão relacionadas ao
contorno < 1 0 3 2 >. A figura \ref{fig:chord-densities-in-time} mostra
acordes com densidades relacionadas ao contorno < 1 0 3 2 >. O contorno
< 1 0 3 2 > tem seu perfil representado graficamente pela figura
\ref{fig:representacao-1032}.

\begin{figure}[h]
  \centering
  \subfloat[Representação gráfica]{
    \includegraphics{c-1032}
    \label{fig:representacao-1032}
  }
  \quad
  \subfloat[Dinâmica]{
    \includegraphics{dynamics-in-time}
    \label{fig:dynamics-in-time}
  }

  \subfloat[Altura]{
    \includegraphics{ly-1032}
    \label{fig:melodias-1032}
  }
  \subfloat[Densidade]{
    \includegraphics{chord-densities-in-time}
    \label{fig:chord-densities-in-time}
  }
  \caption{Contorno < 1 0 3 2 > associado a diferentes parâmetros}
  \label{fig:representacoes}
\end{figure}

O estudo de contornos é importante porque, assim como conjuntos de
notas e motivos, contornos podem ajudar a dar coerência a uma obra
musical \cite{clifford95:contour}. Contornos representam estruturas
musicais manipuláveis através de várias operações como inversão e
retrogradação, e podem ser abordados tanto do ponto de vista da
análise quanto da composição. Contornos já foram usados
satisfatoriamente para analisar obras de Mozart
\cite{beard03:contour}, Schönberg \cite{friedmann85:methodology},
Webern \cite{clifford95:contour,sampaio08:analise}, Dallapicola
\cite{marvin88:generalized} e Reich \cite{quinn97:fuzzy}, e para a
composição musical \cite{sampaio08:em}.

O estudo de contornos tem ainda outros pontos positivos
significativos. Contornos são estruturas de fácil representação
gráfica comprovada por experimentos que mostram que leigos e
músicos---estes últimos com maior acuidade---têm habilidade de usar
desenhos gráficos como o da figura \ref{fig:representacao-1032} para
representar contornos \cite[p. 69]{marvin88:generalized}. Além disso
teorias de contornos fornecem conceitos e operações úteis para
comparação de contornos. Estas operações são definidas matematicamente
de forma semelhante às operações da teoria dos conjuntos.

As teorias de contornos musicais definidas por Friedmann
\cite{friedmann85:methodology}, Morris
\cite{morris87:composition,morris93:directions}, Marvin e Laprade
\cite{marvin.ea87:relating,marvin88:generalized} contêm diversas
declarações matemáticas, como inversão (vide equação
(\ref{eq:1})\footnote{De acordo com esta equação, a inversão de um
  elemento P é a subtração da cardinalidade do contorno por 1 e pelo
  valor do elemento P.}), forma prima e redução de contornos.

\begin{equation}
  \label{eq:1}
  I(P_n) = (q − 1 − P_n)
\end{equation}

Estudos preliminares mostram uma inconsistência entre o algoritmo de
forma prima e as classes de segmentos de contornos definidas por
Marvin e Laprade \cite{marvin.ea87:relating}.

O trabalho inicial destes autores tem sido expandido desde então por
Polansky e Bassein \cite{polansky.ea92:possible}, Clifford
\cite{clifford95:contour}, Quinn \cite{quinn97:fuzzy}, Beard
\cite{beard03:contour}, Bor \cite{bor09:contour} e Schultz
\cite{schultz08:melodic,schultz09:diachronic}. Nenhum destes autores,
entretanto, menciona esta inconsistência.

O conceito de forma prima é bastante importante para estas teorias,
pois estabelece relações entre famílias de contornos. A forma prima é,
então, um pilar das teorias de contornos musicais. O problema de
inconsistência relacionado com a definição de forma prima pode
implicar em fragilidades nas ramificações destas teorias, e nas
análises de contornos de obras musicais publicadas até então.

É necessária uma revisão de todas as declarações destas teorias e suas
ramificações para levantar possíveis inconsistências e propor soluções.

\section{Objetivos}
\label{sec:objetivos}

O principal objetivo do trabalho é levantar possíveis inconsistências
nas teorias de contornos de Friedmann, Morris e Marvin e propor
soluções para tais problemas.

Este trabalho tem como objetivos secundários:

\begin{enumerate}
\item implementar as operações das teorias de contornos no
  \eng{Villa-Lobos Contour Module} \cite{sampaio.ea10:villa-lobos}.
\end{enumerate}

\section{Justificativa}
\label{sec:justificativa}

Diversos trabalhos baseados nas teorias de contornos musicais de
Friedmann, Morris e Marvin, podem conter fragilidades decorrentes do
problema de inconsistência entre o algoritmo de forma prima e tabela
de classes de segmento de contornos de Marvin e Laprade
\cite{marvin.ea87:relating}.

\section{Fundamentação teórica}
\label{sec:fund-teor}

%% inserir bibliografia com relação ferramentas de computação x música

Contornos melódicos são abordados em diversos estudos na literatura
sob o enfoque da Harmonia Tonal \cite{piston59:harmony}, Composição
\cite{schoenberg67:fundamentals,toch77:shaping}, Percepção
\cite{edworthy85:musical,dewitt.ea86:recognition}, Análise
\cite{adams76:melodic,friedmann85:methodology,friedmann87:response,marvin.ea87:relating,marvin88:generalized,marvin91:perception,marvin.ea95:generalization,morris87:composition,morris93:directions,morris95:compositional,clifford95:contour,beard03:contour}
e aplicação de operações de contorno na composição musical
\cite{sampaio08:em}.

%% falar algumas linhas sobre cada autor

%% piston

Piston abordou funções e estruturas da melodia e a sua forma (ou
\eng{shape}) \cite{piston59:harmony}. Afirmou que a distribuição de
notas em uma melodia é marcada por uma variedade de mudanças de
direção, de registro, de pontos graves e agudos, e dos locais onde
estes elementos ocorrem na melodia. Afirmou ainda que contorno é a
reunião de todos estes elementos e que tem importância determinante do
caráter da melodia.

%% inserir schonberg, toch, edworthy e dewitt.ea

%% adams

Adams formalizou um conceito e tipologia de contornos melódicos para
análise descritiva e comparativa \cite{adams76:melodic}. Foram
definidos 15 tipos de contornos a partir de três idéias: a) redução de
uma melodia a apenas quatro pontos---registro da nota inicial, final,
da nota mais aguda e mais grave; b) da relação de registro entre estes
pontos, isto é, se o ponto inicial tem mesmo registro, é mais grave ou
agudo que o ponto final, se o ponto mais agudo ocorre antes ou depois
do ponto final, e ainda se o ponto mais agudo e/ou mais grave
coincidem com os pontos inicial e/ou final; e c) da possível
recorrência entre estes pontos.

%% friedmann

Friedmann desenvolveu operações de contorno como ferramentas para
análise de música do século XX e as aplicou à análise de obras de
Schönberg \cite{friedmann85:methodology}. Ele criou operações como
\eng{Contour Adjacency Series}, \eng{Contour Adjacency Series Vector},
\eng{Contour Class}, \eng{Contour Interval}, \eng{Contour Interval
  Sucession}, \eng{Contour Interval Array}, e \eng{Contour Class
  Vector}\footnote{Estas operações estão amplamente explicadas no
  artigo de Friedmann e nas revisões de literatura de
  contornos.}. Friedmann ainda abordou o problema da terminologia em
operações de contorno \cite{friedmann87:response} e serviu de ponto de
partida para o trabalho de Marvin \cite{marvin88:generalized}.

%% morris

Morris publicou um manual de técnicas composicionais do século XX no
qual discutiu espaços musicais como espaço de altura, e de contorno
\cite{morris87:composition}. Nesta publicação ele formalizou
representação numérica de contornos, matriz de comparação. Expandiu
ainda seu trabalho com um algoritmo para redução de contornos
\cite{morris93:directions}.

%% marvin

O trabalho de Marvin tem como ponto de partida as relações de
semelhança de espaço de contorno, definido por Morris. Ela desenvolveu
em co-autoria com Laprade \cite{marvin.ea87:relating} um algoritmo
para obtenção de forma prima, uma tabela de classes de contorno
(semelhante à tabela de Allen Forte, na teoria dos conjuntos),
estabeleceu medidas de similaridade tanto para contornos de mesma
cardinalidade quanto de cardinalidade diferente e testou estas
ferramentas em análise de obras. Além disso Marvin abordou relações
entre modelos analíticos e composicionais de estruturas musicais e a
capacidade auditiva de ouvintes \cite{marvin88:generalized}. De acordo
com ela teorias abstratas de estruturas de classes de alturas e
conjuntos não refletem a percepção auditiva dos ouvintes tão bem
quanto teorias que modelam a articulação destas estruturas de classes
de alturas e conjuntos \cite[p. 228]{marvin88:generalized}. Ouvintes
costumam criar representações gráficas do contexto musical, como
mudanças de direção da melodia, padrões de duração relativa e assim
por diante. Estas representações podem ser comparadas umas às outras a
partir da aplicação e generalização das teorias de contornos de Morris
\cite[p. 229]{marvin88:generalized}. Marvin propôs ainda generalização
de contornos para outros domínios e aplicou esta idéia ao espaçamento
de acordes na obra pianística de Luigi Dallapicola. Marvin propôs como
trabalho futuro na área de psicologia musical experimentos que
apliquem a sua precisa definição de equivalência de contornos. Por fim
propôs um currículo para pedagogia de teoria para música não tonal.

%% inserir polansky, quinn

%% clifford

Clifford investigou o uso de relações de contorno como elemento
estrutural em obras pré-seriais de Anton Webern
\cite{clifford95:contour}. Concluiu em seu trabalho que na ausência de
outros elementos de organização de alturas, contorno representa um
fator estrutural igual em significado a relações de alturas ou de
classes de conjuntos. Verificou a existência de três níveis de
contornos: a) contorno no nível melódico, b) contorno como delineador
da forma, que ocorre em larga escala, e c) contorno textural, que
envolve contornos distribuído em vozes diferentes. Além disso propôs
trabalhos futuros como investigação de contornos em outras obras de
Webern, exploração do contorno textural em múltiplos níveis, relações
entre percepção auditiva e imagens dos contornos, e por fim exploração
de contornos associados a outros parâmetros além de altura.

%% beard

Beard implementou a duração na análise a partir de contornos melódicos
através do princípio estatístico de múltiplas regressões lineares
\cite{beard03:contour}. Aplicou com êxito este princípio na análise
dos primeiros temas de 19 sonatas de Mozart. Beard sugere como
trabalho futuro a análise de estruturas de outras partes das obras de
Mozart ou ainda das obras de outros compositores, experimentos com as
técnicas de regressão e estudos de percepção auditiva para investigar
a relação entre percepção e regressão.

%% minha dissertação

Em minha dissertação experimentei funcionalidades das operações de
contorno usadas para análise na composição da obra \obra{}
\cite{sampaio08:em}. Compus a maior parte do material da obra a partir
de operações aplicadas a um único contorno. Embora tenha concentrado o
estudo em contornos associados à altura, experimentei associações com
andamento, densidade e textura. Além disso iniciei o desenvolvimento
do software \goiaba{}, para processamento de contornos.

%% inserir bohr e schultz

\section{Procedimentos metodológicos}
\label{sec:metodologia}

O levantamento das inconsistências pode ter precisão e eficácia
elevados com o uso de ferramentas computacionais. Por esta razão este
trabalho demanda do desenvolvimento de um programa de computador
desenhado para abrigar funções e testes para todas as definições das
teorias de contorno musical de Friedmann, Morris e Marvin, e das suas
ramificações.

Tal programa deve ser desenvolvido segundo uma abordagem \eng{bottom
  up} \cite{graham94:lisp} de forma que possa ser desenvolvido e
expandido à medida que se avança na programação das definições das
teorias de contornos musicais.

%% inserir etapas:

%% levantamento das operações definidas na literatura
%% levantamento das operações utilizadas em análises na literatura
%% catalogação dos exemplos utilizados na literatura
%% criação de testes funcionais com exemplos usados na literatura

\section{Resultados esperados}
\label{sec:resultados-esperados}

Espera-se com este trabalho levantar e propor soluções para as
possíveis fragilidades nas definições teorias de contornos musicais e

\section{Estrutura de tópicos da tese}
\label{sec:estrutura-de-topicos}

\begin{itemize}
\item Introdução
\item Revisão de literatura (comparação de teorias e sondagem do
  estado de arte)
\item Apresentação das ferramentas computacionais
\item Discussão de como as ferramentas ajudam a encontrar os problemas
  das teorias de contornos.
\item Propostas para solução de problemas das teorias de contornos.
\end{itemize}

\section{Cronograma}
\label{sec:cronograma}

\begin{table}[htbp]
  \small
  \centering
  \begin{tabular}{|l||c|c|c|c||c|c|c|c|}
    \hline
    & \multicolumn{4}{|c||}{2011} & \multicolumn{4}{|c||}{2012} \\
    \hline
    \hline
    \textbf{Atividades}
  \end{tabular}
  \caption{Cronograma das atividades (em trimestres)}
  \label{tab:cronograma}
\end{table}

\nocite{marvin.ea95:generalization,sampaio.ea10:villa-lobos,sampaio.ea09:goiaba,sampaio.ea09:musical}

\renewcommand{\refname}{Bibliografia}
\bibliography{bib,genos}
\bibliographystyle{plain}

\end{document}